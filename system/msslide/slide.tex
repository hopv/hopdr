\documentclass{jarticle}

\usepackage{amsmath, amssymb}
\usepackage{mathtools}
\usepackage{braket}
\usepackage{stmaryrd}
\usepackage{tikz}
\usepackage{mathpartir}
\usepackage{pifont}
\usepackage{multirow}
\usepackage{algorithmic}
\usepackage{graphicx}
\usepackage{color}
%\usepackage{xcolor}
\usepackage{amsthm}
%\usepackage{bcprules}

\usepackage{algorithm,algorithmic}

\newcommand\todo[1]{\textcolor{green}{[TODO: #1]}}

\definecolor{mygray}{gray}{0.45}

% names
\newcommand \nhz{\(\nu\text{HFL}_\mathbb{Z}\)}
\newcommand \hflz {\(\text{HFL}_\mathbb{Z}\)}
\newcommand \pahfl{\textsc{PaHFL}}
\newcommand \rethfl{{\sc ReTHFL}}
\newcommand \horus{Horus}
\newcommand \hopdr {{\sc HoPDR}}



\newcommand\COL{\mathbin{:}}

\theoremstyle{definition}
\newtheorem{definition}{Definition}
\newtheorem{example}{Example}
\newtheorem{lemma}{Lemma}
\newtheorem{conjecture}{Conjecture}
\newtheorem{theorem}{Theorem}

% general math
\DeclareMathOperator{\dom}{dom}
\DeclareMathOperator{\fv}{fv}
\newcommand \gfp {\textbf{gfp}}

% nhz syntax
\newcommand \formula {\varphi}
\newcommand \goal {\psi}
\newcommand \true {\textbf{tt}}
\newcommand \false {\textbf{ff}}
\newcommand \predicate {\textbf{p}}
\newcommand \arith {\textbf{a}}
\newcommand \operator {\mathbin{\textbf{op}}}
\newcommand \term {M}
\newcommand \atom {A}
\newcommand \definite {D}

% nu HFLz semantics
%\newcommand {\val}[1]{[\![#1]\!]}
\newcommand {\val}[1]{ \llbracket #1 \rrbracket }
\newcommand {\vwt}[3][]{\val{\wt{#2}{#3}}(\alpha\ifthenelse{\isempty{#1}}{}{[#1]})}
\newcommand {\wta}[3]{\Gamma, #1 \vdash_H #2 : #3}
\newcommand {\vwta}[3]{\val{\wta{#1}{#2}{#3}}(\alpha)}

\newcommand {\valfun}{\alpha}
\newcommand {\domain}[1]{\mathcal{D}_{#1}}
\newcommand {\order}[1]{\sqsubseteq_{#1}}
\newcommand {\join}[1]{\sqcap_#1}
\newcommand {\meet}[1]{\sqcup_#1}
\newcommand {\tohfl}{\mathit{toHFL}}
% nu hflz operator
\newcommand {\fE} {\Gamma^f}
\newcommand {\clauses} {\mathcal{C}}
\newcommand {\system} {\mathcal{S}}
\newcommand {\fEo} {\sqsubseteq}

% simple type
\newcommand \stypeint {\textbf{Int}}
\newcommand \stypebool {\bullet}
\newcommand \stypeboolf {\circ}
\newcommand \srtype {\rho} % simple result type
\newcommand \satype {\eta} % simple argument type
\newcommand \senv {\Delta}
\newcommand \stE {\senv}
\newcommand \stypes {\vdash_H}
\newcommand \wt[3][]{\ifthenelse{\isempty{#1}}{\senv}{\senv, #1} \vdash_H #2: #3}

% refinement type syntax
\newcommand \constraint {\theta}
\newcommand \refty {\tau}
\newcommand \ty {\refty}
\newcommand \typeint[1]{{#1} : \stypeint}
\newcommand \typebool[1]{\stypebool \langle #1 \rangle}
\newcommand \typeboolf[1]{\stypeboolf \langle #1 \rangle}
\newcommand \ti[1]{\typeint{#1}}
\newcommand \tb[1]{\typebool{#1}}
\newcommand \taus{\{\, \tau_1, \dots, \tau_n \,\}}

% refinement type judgement
\newcommand \refines[2]{#1:: #2}
\newcommand \tenv {\Gamma}
\newcommand \rtE {\tenv}
\newcommand \wf[2]{\tenv \vdash \refines{#1}{#2}}
\newcommand \minimalize[1] {#1_{\Downarrow}}
\newcommand \intE {\mathcal{I}}

% refinement type semantics
%\newcommand {\semt}[1]{(\!|#1|\!)}
\newcommand {\semt}[1]{\llparenthesis #1 \rrparenthesis}
\newcommand {\mjudge}[2]{\tenv \models #1 \COL #2}
\newcommand {\semsbt}[3][]{\tenv; \ifthenelse{\isempty{#1}}{\dcnstr}{#1} \models #2 \subtypeRelation #3}
\newcommand {\wsemsbt}[4][]{\tenv; \ifthenelse{\isempty{#1}}{\dcnstr}{#1} \models #2 \subtypeRelation_{#4} #3}
\newcommand {\srt}[2]{\semt{\wf{#1}{#2}}(\alpha)}
\newcommand {\srtg}[3][]{\semt{\wf{#2}{#3}}(\alpha[#1])}
\newcommand {\vc}[1]{[\!|#1|\!]}
\newcommand {\vtenv}[2][]{\val{#2}({\ifthenelse{\isempty{#1}}{\valfun}{#1}})}
%\newcommand {\semttaus}[1][] { \bigcap_i \semt{\tau_i}({\ifthenelse{\isempty{#1}}{\valfun}{#1}})}
\newcommand {\semttaus} {\bigcap_i \semt{\tau_i}(\valfun)}


% PDR rules
\newcommand {\rdecide} {\emph{Decide}}
\newcommand {\rconflict} {\emph{Conflict}}
\newcommand {\rvalid} {\emph{Valid}}
\newcommand {\rinvalid} {\emph{Invalid}}
\newcommand {\rcandidate} {\emph{Candidate}}
\newcommand {\runfold} {\emph{Unfold}}
\newcommand {\rinitialize} {\emph{Initialize}}
\newcommand {\rinduction} {\emph{Induction}}
\newcommand {\runknown} {\emph{Unknown}}

\newcommand {\invalid} {\textbf{Invalid}}
\newcommand {\valid} {\textbf{Valid}}
\newcommand {\unknown} {\textbf{Unknown}}

\newcommand {\pdrmid} {\parallel}
\newcommand {\pdrapp} {\Longrightarrow }
\newcommand {\Tr} {\pdrapp}
\newcommand {\pdrtop} {\psi_{\mathcal{G}}}
\newcommand {\pdrtransform} {\mathcal{F}}
\newcommand {\pdrF} {\pdrtransform}
\newcommand {\floor}[1] {\lfloor #1 \rfloor }
\newcommand {\pdrtypes} {\uparrow}
\newcommand {\T} {\top_{\stE}}
\newcommand {\F} {\bot_{\stE}}
\newcommand {\E} {\rtE}
\newcommand {\PT} {\pdrtop}

\newcommand {\cex} {\mathcal{C}}
\newcommand {\aprx} {\mathcal{A}}

% negative types
\newcommand {\fenv} {\Delta} % negative type environment
\newcommand {\semtf}[1]{(\!|#1|\!)_\circ}
\newcommand {\srtf}[2]{\semtf{\wf{#1}{#2}}(\alpha)}

% conflict relation
\newcommand {\conflict}{\not \sim}
\newcommand {\consistent}{\sim}

\newcommand {\intersect} {\land}
\newcommand {\interpolation} {\textbf{Interp}}

% dual
\newcommand {\dual}[1]{\overline{#1}}

% non-idempotent intersection types
\newcommand \itypes {\vdash_I}
\newcommand \isBE {\mathit{isBaseEnv}} % isBaseEnv
\newcommand \sty {\mathit{sty}} % sty
\newcommand \iE {\tilde{\Delta}}
\newcommand \ity {\tilde{\srtype}}
\newcommand \tmpgen {\mathit{TempGen}}
\newcommand \newpred {\mathit{NewPred}}
\newcommand \ertE {\tilde{\rtE}}
\newcommand \erty {\tilde{\refty}}
\newcommand \econstr {\Theta}


\begin{document}

\begin{align*}
    &\forall m.\: m \leq 0 \vee
    (\nu X.\: \lambda x.\: x \neq 0
        \wedge X\ (x + 1))\ m \\
    &\Leftrightarrow \nu X.\: \lambda x.\: x \neq 0 \wedge X\ (x + 1)
\end{align*}

\begin{align*}
    \Rightarrow
\end{align*}

\begin{align*}
    \begin{cases}
        \forall m.\: m > 0 \Rightarrow R(m) \\
        R(x) \Leftrightarrow x \neq 0 \wedge R(x+1)
    \end{cases}
\end{align*}

\begin{align*}
  (\nu X.\: \lambda x.\: x \neq 0 \wedge X\ (x + 1))\ m
  &= m \neq 0 \wedge (\nu X.\: \lambda x.\: x \neq 0 \wedge X\ (x + 1))\ (m + 1)
    \\
  &= m \neq 0 \wedge m +1 \neq 0 \wedge (\nu X.\: \lambda x.\: x \neq 0 \wedge
          X\ (x + 1))\ (m + 2) \\
  &\quad \vdots \\
  &= m \neq 0 \wedge m +1 \neq 0 \wedge \dots \wedge m + k \neq 0 \cdots \\
  &\quad \vdots \\
\end{align*}

\begin{align*}
    X(x) \Leftrightarrow x \neq 0 \wedge X(x+1)
\end{align*}
\begin{align*}
  &(\nu X.\: \lambda x.\: x \neq 0 \wedge X\ (x + 1))\ m &&\\
  &= m \neq 0 \wedge (\nu X.\: \lambda x.\: x \neq 0 \wedge X\ (x + 1))\ (m + 1)
    &&m \neq 0 \\
  &= m \neq 0 \wedge m +1 \neq 0 \wedge (\nu X.\: \lambda x.\: x \neq 0 \wedge
          X\ (x + 1))\ (m + 2)  && m \neq 0 \wedge m + 1 \neq 0 \\
  &\quad \vdots &&\\
  &= m \neq 0 \wedge m +1 \neq 0 \wedge \dots \wedge m + k \neq 0 &&  \\
  &\quad (\nu X.\: \lambda x.\: x \neq 0 \wedge X\ (x + 1))\ (m + k + 1) &&
  m \not \in \{\, 0, \dots, -k \,\} \\
  &\quad \vdots &&\\
\end{align*}

\begin{align*}
  &(\nu X.\: \lambda x.\: x \neq 0 \wedge X\ (x + 1))\ m &&\\
  &= m \neq 0 \wedge (\nu X.\: \lambda x.\: x \neq 0 \wedge X\ (x + 1))\ (m + 1)
    &&m \neq 0 \\
  &= m \neq 0 \wedge m +1 \neq 0 \wedge (\nu X.\: \lambda x.\: x \neq 0 \wedge
          X\ (x + 1))\ (m + 2)  && m \neq 0 \wedge m + 1 \neq 0 \\
  &\quad \vdots &&\\
  &= m \neq 0 \wedge m +1 \neq 0 \wedge \dots \wedge m + k \neq 0 &&  \\
  &\quad (\nu X.\: \lambda x.\: x \neq 0 \wedge X\ (x + 1))\ (m + k + 1) &&
  m \not \in \{\, 0, \dots, -k \,\} \\
  &\quad \vdots &&\\
\end{align*}

\begin{align*}
    \goal \pdrtypes \refty
\end{align*}
\begin{align*}
    \floor{\refty}
\end{align*}

\begin{align*}
    \lambda x.\: x \neq 0 \pdrtypes \ti{x} \to \tb{x > 0}
    &= \forall x.\: x > 0 \Rightarrow x \neq 0
\end{align*}
\begin{align*}
    \lambda p.\: p\ 0 \pdrtypes (\ti{x} \to \tb{x \geq 0}) \to \tb{\true}
    &= \true \Rightarrow (\lambda x\:. x \geq 0)\ 0
\end{align*}

\begin{align*}
    \floor{\ti{x} \to \tb{x \geq 0}} = \lambda x\:. x \geq 0
\end{align*}

\begin{align*}
    \floor{(\ti{x} \to \tb{x \geq 0}) \to \tb{\true}}
        &= \lambda p.\: (p \pdrtypes \ti{x} \to \tb{x \geq 0}) \wedge \true \\
        &= \lambda p.\: \forall x.\: x \geq 0 \Rightarrow p\ x
\end{align*}

\begin{align*}
  &(\nu X.\: \lambda x.\: x \neq 0 \wedge X\ (x + 1))\ m \\
  &= m \neq 0 \wedge (\nu X.\: \lambda x.\: x \neq 0 \wedge X\ (x + 1))\ (m + 1)
    \\
  &= m \neq 0 \wedge m +1 \neq 0 \wedge (\nu X.\: \lambda x.\: x \neq 0 \wedge
          X\ (x + 1))\ (m + 2) \\
  &\quad \vdots \\
\end{align*}

\begin{align*}
  &{ \color{red} (\nu X.\: \lambda x.\: x \neq 0 \wedge X\ (x + 1))} \ m \\
  &= m \neq 0 \wedge { \color{red} (\nu X.\: \lambda x.\: x \neq 0 \wedge X\
          (x + 1))}\ (m + 1)
    \\
  &= m \neq 0 \wedge m +1 \neq 0 \wedge {\color{red}(\nu X.\: \lambda x.\: x \neq 0 \wedge
          X\ (x + 1))}\ (m + 2) \\
  &\quad \vdots \\
\end{align*}

\begin{align*}
  &0.\: \true \\
  &1.\: m \neq 0 \wedge \true \\
  &2.\: m \neq 0 \wedge m +1 \neq 0 \wedge \true \\
  &\quad \vdots \\
\end{align*}

\begin{align*}
    \pdrF(R)(x) = x \neq 0 \wedge R(x+1)
\end{align*}

\begin{align*}
    &\pdrF(\true)(x) = x \neq 0 \\
    &\pdrF^2(\true)(x) = \pdrF(x\neq0)(x) = x \neq 0 \wedge x + 1 \neq 0
\end{align*}

\begin{align*}
    \bigwedge_{i > 0} \pdrF(\true)
\end{align*}

\begin{align*}
    \pdrF(R) \Leftrightarrow R
\end{align*}

\begin{align*}
    \begin{cases}
        \forall m.\: m > 0 \Rightarrow R(m) \\
        \pdrF(R) \Leftrightarrow R
    \end{cases}
\end{align*}

\begin{align*}
    \forall m.\: m > 0 \Rightarrow R(m) \\
\end{align*}

\begin{align*}
    &\pdrF^0(\true)(x) = \true \\
    &\pdrF(\true)(x) = x \neq 0 \\
    &\pdrF^2(\true)(x) = x \neq 0 \wedge x + 1 \neq 0
\end{align*}

\begin{align*}
    \begin{cases}
        { \color{red} \forall m.\: m > 0 \Rightarrow R(m) }\\
        \pdrF(R) \Leftrightarrow R
    \end{cases}
\end{align*}

\begin{align*}
    \forall m.\: m > 0 \Rightarrow \false  \quad \Leftrightarrow \quad \false
\end{align*}

\begin{align*}
    \pdrF(R_1) = \pdrF(x \neq 0) = x \neq 0 \wedge x + 1 \neq 0
\end{align*}

\begin{align*}
    \begin{cases}
        \pdrF(R_1) \Leftarrow \varphi \\
        \forall m.\: m > 0 \Rightarrow \varphi(m)
    \end{cases}
\end{align*}

\begin{align*}
x > 0
\end{align*}

\begin{align*}
    \pdrF(\varphi) &= [\varphi/\text{Sum}]\goal_{\text{sum}}
\end{align*}
\begin{align*}
    \begin{cases}
        [\floor{\tau}/\text{Sum}]\pdrtop \\
        \floor{\tau} \pdrtypes \tau
    \end{cases}
\end{align*}

\begin{align*}
    \goal_{\text{sum}} &::= \lambda x.\: \lambda k.\: (x > 0 \lor k\ x) \land \\
                     &\quad\quad  (x \leq 0 \lor \text{Sum}\ (x - 1)\ (\lambda r. k\ (x + r)).\\
    \pdrtop &::= \forall n. \text{Sum}\ n\ (\lambda r. r \geq n).
\end{align*}

\begin{align*}
    &\pdrF(\floor{\ti{x} \to (\ti{y} \to \tb{\false}) \to \tb{\true}}) \\
    &=\pdrF(\lambda x.\: \lambda k.\: \true) \\
    &= \lambda x.\: \lambda k.\: (x > 0 \lor k\ x)
\end{align*}

\begin{align*}
    \forall n.\: (\nu \text{Sum}.\: \psi_{\text{sum}})\ n\ (\lambda r. r \geq n)
\end{align*}

\begin{align*}
    \ti{x} \to (\ti{y} \to \tb{\false}) \to \tb{\true}
\end{align*}

\begin{align*}
    \ti{x} \to (\ti{y} \to \tb{\true}) \to \tb{\false}
\end{align*}

\begin{align*}
{\color{red} \ti{x} \to (\ti{y} \to \tb{\true}) \to \tb{\false}}
\end{align*}
\begin{align*}
    \begin{cases}
{\color{red} [\floor{\tau}/\text{Sum}]\pdrtop} \\
        \pdrF(\floor{\tau}) \pdrtypes \tau
    \end{cases}
\end{align*}
\begin{align*}
        [\floor{\tau_1}/\text{Sum}]\pdrtop = \forall n.\: \false\\
\end{align*}

\begin{align*}
    [\pdrF(\floor{\tau_0})/\text{Sum}]\pdrtop = \forall n.\: \true
\end{align*}

\begin{align*}
    \begin{cases}
        [\floor{\tau}/\text{Sum}]\pdrtop \\
        \pdrF(\floor{\tau_0}) \pdrtypes \tau
    \end{cases}
\end{align*}

\begin{align*}
    \forall m.\: m > 0 \Rightarrow (\nu X.\: x\neq 0 \wedge X(x + 1))\ m
\end{align*}

\begin{align*}
    \begin{cases}
        [\floor{\tau}/\text{Sum}]\pdrtop \\
        \pdrF(\floor{\tau}) \pdrtypes \tau
    \end{cases}
\end{align*}

\begin{align*}
&{ \color{red} (\nu X.\: \lambda x.\: x \neq 0 \wedge X\ (x + 1)) \ m} \\
  &= m \neq 0 \wedge { \color{red} (\nu X.\: \lambda x.\: x \neq 0 \wedge X\
          (x + 1))\ (m + 1)}
    \\
  &= m \neq 0 \wedge m +1 \neq 0 \wedge {\color{red}(\nu X.\: \lambda x.\: x \neq 0 \wedge
          X\ (x + 1))\ (m + 2)} \\
  &\quad \vdots \\
\end{align*}

\begin{align*}
    &{\color{red} (\nu X.\: \lambda x.\: x \neq 0 \wedge X\ (x + 1))}\ m \\
    &= m \neq 0 \wedge {\color{red} (\nu X.\: \lambda x.\: x \neq 0 \wedge X\ (x
    + 1))}\ (m + 1)
    \\
    &= m \neq 0 \wedge m +1 \neq 0 \wedge {\color{red} (\nu X.\: \lambda x.\: x
    \neq 0 \wedge
          X\ (x + 1))}\ (m + 2) \\
  &\quad \vdots \\
  &= m \neq 0 \wedge m +1 \neq 0 \wedge \dots \wedge m + k \neq 0 \cdots \\
  &\quad \vdots \\
\end{align*}

\begin{align*}
    &{\color{blue}\forall n.\:} (\nu \text{Sum}.\: {\color{red} \lambda x.\:
    \lambda k.\: } \\
                     &\quad\quad\quad {\color{red}(x > 0 \lor k\ x) \land} \\
                     &\quad\quad\quad  {\color{red}(x \leq 0 \lor \text{Sum}\ (x - 1)\ (\lambda
                     r. k\ (x + r)))}\\
    &\quad )\ {\color{blue}n\ (\lambda r.\: r \geq n)}
\end{align*}
\begin{align*}
    \goal_{\text{sum}} &::= {\color{red}\lambda x.\: \lambda k.\:} \\
    &\quad\quad\quad{\color{red}(x > 0 \lor k\ x) \land} \\
                     &\quad\quad\quad  {\color{red} (x \leq 0 \lor \text{Sum}\ (x - 1)\ (\lambda
                     r. k\ (x + r)))}.\\
    \pdrtop &::= {\color{blue}\forall n. \text{Sum}\ n\ (\lambda r. r \geq n)}.
\end{align*}

\begin{align*}
    P(m) \equiv (\nu X. \lambda x.\: x \neq 0 \wedge X(x + 1))\ m
\end{align*}
\begin{align*}
    { \color{red} P(m) } &\quad \equiv \quad m \neq 0 \wedge {\color{red} P(m+1)}
    \\
    &\quad \equiv \quad m \neq 0 \wedge m +1 \neq 0 \wedge {\color{red} P(m+2)} \\
  &\quad \quad \vdots \\
        &\quad \equiv \quad m \neq 0 \wedge m +1 \neq 0 \wedge \dots \wedge m + k \neq 0  \wedge
        {\color{red} P(m+k+1) }\\
  &\quad \quad \vdots \\
\end{align*}


%\begin{algorithm}[tb]
%\begin{algorithmic}[1]
%    \STATE{\( \tau_1, \dots, \tau_n\) are approximations}
%    \LOOP
%        \IF{\( \tau_n \) is safe and inductive\\ ( \([\floor{\tau_n}/Sum]\pdrtop\)
%        and \( \pdrF(\floor{\tau_n}) \pdrtypes \tau_n\))}
%            \STATE {return \textit{Valid}}
%        \ELSIF{\( \tau_n \) is not safe}
%            \STATE {refine approx by \( \pdrF(\tau_i) \) and \( \pdrtop \)}
%        \ELSE
%            \STATE{expand fixpoint once}
%        \ENDIF
%
%    \ENDLOOP
%\end{algorithmic}
%\end{algorithm}

\begin{algorithm}[tb]
\begin{algorithmic}[1]
    \STATE{\( R_1, \dots, R_n\)が近似列}
    \LOOP
        \IF{\( R_n \) が帰納的不変条件}
            \STATE {問題が\textit{Valid}}
        \ELSIF{\( R_n \) が安全でない}
            \STATE {\( \pdrF(R_i) \)と反例のはさみうちで精緻化}
        \ELSE
            \STATE{不動点を1回展開, 近似列を拡張}
        \ENDIF

    \ENDLOOP
\end{algorithmic}
\end{algorithm}

\begin{algorithm}[tb]
\begin{algorithmic}[1]
    \STATE{\( \tau_1, \dots, \tau_n\) が近似列}
    \LOOP
        \IF{\( \tau_n \) 帰納的不変条件\\ ( \([\floor{\tau_n}/Sum]\pdrtop\)
        and \( \pdrF(\floor{\tau_n}) \pdrtypes \tau_n\))}
            \STATE {問題が\textit{Valid}}
        \ELSIF{\( \tau_n \) が安全でない}
            \STATE {\( \pdrF(\tau_i) \)と反例のはさみうちで精緻化した型を推論}
        \ELSE
            \STATE{不動点を1回展開, 近似列を拡張}
        \ENDIF

    \ENDLOOP
\end{algorithmic}
\end{algorithm}

\end{document}

