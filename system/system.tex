\documentclass[runningheads]{llncs}
\usepackage{amsmath, amssymb}
\usepackage{mathtools}
\usepackage{braket}
\usepackage{stmaryrd}
\usepackage{tikz}
\usepackage{mathpartir}
\usepackage{pifont}
\usepackage{multirow}
\usepackage{algorithmic}
\usepackage{graphicx}
\usepackage{color}
\usepackage{bcprules}

%\usepackage{xcolor}
%\pagecolor[rgb]{0,0,0} %black
%\color[rgb]{0.5,0.5,0.5} %grey


\newcommand \nhz{\(\nu\text{HFL}_\mathbb{Z}\)}
\newcommand \hflz {\(\text{HFL}_\mathbb{Z}\)}

\newcommand\COL{\mathbin{:}}

% nhz syntax
\newcommand \true {\textbf{tt}}
\newcommand \false {\textbf{ff}}
\newcommand \predicate {\textbf{p}}
\newcommand \arith {\textbf{a}}
\newcommand \operator {\mathbin{\textbf{op}}}
\newcommand \term {M}
\newcommand \atom {A}
\newcommand \goal {G}
\newcommand \definite {D}

% nu HFLz semantics
\newcommand {\val}[1]{[\![#1]\!]}
\newcommand {\vwt}[3][]{\val{\wt{#2}{#3}}(\alpha\ifthenelse{\isempty{#1}}{}{[#1]})}
\newcommand {\wta}[3]{\Gamma, #1 \vdash_H #2 : #3}
\newcommand {\vwta}[3]{\val{\wta{#1}{#2}{#3}}(\alpha)}

\newcommand {\valfun}{\alpha}
\newcommand {\domain}[1]{\mathcal{D}_{#1}}
\newcommand {\order}[1]{\sqsubseteq_{#1}}
\newcommand {\join}[1]{\sqcap_#1}
\newcommand {\meet}[1]{\sqcup_#1}
\newcommand \gfp {\textbf{gfp}}


% simple type
\newcommand \stypeint {\textbf{Int}}
\newcommand \stypebool {\bullet}
\newcommand \stypeboolf {\circ}
\newcommand \srtype {\rho} % simple result type
\newcommand \satype {\eta} % simple argument type
\newcommand \senv {\Gamma}
\newcommand \wt[3][]{\ifthenelse{\isempty{#1}}{\senv}{\senv, #1} \vdash_H #2: #3}

% refinement type syntax
\newcommand \constraint {\theta}
\newcommand \refty {\tau}
\newcommand \typeint[1]{{#1} : \stypeint}
\newcommand \typebool[1]{\stypebool \langle #1 \rangle}
\newcommand \typeboolf[1]{\stypeboolf \langle #1 \rangle}

% refinement type judgement
\newcommand \refines[2]{#1:: #2}
\newcommand \tenv {\Gamma}
\newcommand \wf[2]{\tenv \vdash \refines{#1}{#2}}
\newcommand \minimalize[1] {#1_{\Downarrow}}

% refinement type semantics
\newcommand {\semt}[1]{(\!|#1|\!)_\bullet}
\newcommand {\mjudge}[2]{\tenv \models #1 \COL #2}
\newcommand {\semsbt}[3][]{\tenv; \ifthenelse{\isempty{#1}}{\dcnstr}{#1} \models #2 \subtypeRelation #3}
\newcommand {\wsemsbt}[4][]{\tenv; \ifthenelse{\isempty{#1}}{\dcnstr}{#1} \models #2 \subtypeRelation_{#4} #3}
\newcommand {\srt}[2]{\semt{\wf{#1}{#2}}(\alpha)}
\newcommand {\srtg}[3][]{\semt{\wf{#2}{#3}}(\alpha[#1])}
\newcommand {\vc}[1]{[\!|#1|\!]}
\newcommand {\vtenv}[2][]{\val{#2}({\ifthenelse{\isempty{#1}}{\valfun}{#1}})}

% PDR rules
\newcommand {\rdecide} {\emph{Decide}}
\newcommand {\rconflict} {\emph{Conflict}}
\newcommand {\rvalid} {\emph{Valid}}
\newcommand {\rcandidate} {\emph{Candidate}}
\newcommand {\runfold} {\emph{Unfold}}
\newcommand {\rinitialize} {\emph{Initialize}}

% negative types
\newcommand {\fenv} {\Delta} % negative type environment
\newcommand {\semtf}[1]{(\!|#1|\!)_\circ}
\newcommand {\srtf}[2]{\semtf{\wf{#1}{#2}}(\alpha)}

% conflict relation
\newcommand {\conflict}{\not \sim}
\newcommand {\consistent}{\sim}

\newcommand {\intersect} {\land}
\newcommand {\interpolation} {\textbf{Interp}}

% dual
\newcommand {\dual}[1]{\overline{#1}}

\newlength\myindent
\setlength\myindent{2em}
\newcommand\bindent{%
  \begingroup
  \setlength{\itemindent}{\myindent}
  \addtolength{\algorithmicindent}{\myindent}
}
\newcommand\eindent{\endgroup}


\begin{document}

\section{Higher-order PDR}

The target problem is \nhz{} validity checking.
For the syntax of the target language, we adopt the following HES-like language,
which is equivalent to \nhz{} since we can translate them mutually.

\subsection{Target Language}

\begin{align*}
    \goal &\quad ::= \quad
          G \land G
          \mid G \lor G
          \mid \forall x. G
          \mid \atom
          \mid \constraint
         \\
    \definite &\quad ::= \quad X x_1 \cdots x_n =_\nu G
          \\
    \atom &\quad ::= \quad X \mid A_1\ A_2\\
    \constraint &\quad ::= \quad \cdots
\end{align*}


Then \nhz{} validity checking is that given definite formulas
\( X_1 \bar{x_1} = \psi_1, \cdots X_n \bar{x_n} =  \psi_n \)
and a goal formula \( \psi \),
we check the validity of \( \psi \).

\subsection{refinement type}

\subsubsection{Syntax}

\begin{align*}
  &\textit{arithmetic expressions} &
  \arith &::=
  n
  \mid
  x
  \mid
  \operator(\arith_1, \cdots, \arith_n)
  \\
  &\textit{constraint formulas}&
  \constraint &::=
  \true
  \mid \false
  \mid \predicate(\arith_1, \cdots, \arith_n)
  \mid \constraint_1 \land \constraint_2
  \mid \constraint_1 \lor \constraint_2
  \\
  &\textit{positive types}&
  \refty &::=
  \typebool{\theta}
  \mid t \rightarrow \refty_2
  \\
  &&
  t &::=
  \tau
  \mid \typeint{x}
  \mid t_1 \land t_2
  \\
  &\textit{negative types}&
  \sigma &::=
  \typeboolf{\theta}
  \mid \typeint{x} \rightarrow \sigma
  \mid \sigma_1 \rightarrow \sigma_2
  %\mid \sigma_1 \intersect \sigma_2
\end{align*}

\subsubsection{Semantics}

\begin{align*}
    \srt{\typebool{\constraint}}{\stypebool} &= \{\val{\tenv \vdash_c \constraint}(\alpha), \top \}\\
    \srt{\typeint{x} \rightarrow \refty}{\stypeint \rightarrow \srtype} &=
      \{f \in \domain{\stypeint \rightarrow \srtype} \mid \forall v \in \domain{\stypeint}. \\
          &\quad \quad f(v) \in \semt{\tenv, x \vdash \refty:: \srtype}(\alpha[x \mapsto v])\\
    \srt{\refty_1 \rightarrow \refty_2}{\srtype_1 \rightarrow \srtype_2} &= \{f \in \domain{\srtype_1 \rightarrow \srtype_2} \mid \forall v \in
        \srt{\refty_1}{\srtype_1}. \\
          &\quad \quad \quad f(v) \in \srt{\refty_2}{\srtype_2}\}\\
    \srt{\refty_1 \intersect \refty_2}{\srtype} &=
    \srt{\refty_1}{\srtype} \cap \srt{\refty_2}{\srtype}
\end{align*}
and
\begin{align*}
    \srtf{\typeboolf{\constraint}}{\stypebool} &= \{\val{\tenv \vdash_c \lnot \constraint}(\alpha), \bot \}\\
    \srtf{\typeint{x} \rightarrow \sigma}{\stypeint \rightarrow \srtype} &=
      \{f \in \domain{\stypeint \rightarrow \srtype} \mid \forall v \in \domain{\stypeint}. \\
          &\quad \quad f(v) \in \semtf{\tenv, x \vdash \sigma :: \srtype}(\alpha[x \mapsto v])\\
    \srtf{\sigma_1 \rightarrow \sigma_2}{\srtype_1 \rightarrow \srtype_2} &= \{f \in \domain{\srtype_1 \rightarrow \srtype_2} \mid \forall v \in
        \srtf{\sigma_1}{\srtype_1}. \\
          &\quad \quad \quad f(v) \in \srtf{\sigma_2}{\srtype_2}\}\\
    \srtf{\sigma_1 \intersect \sigma_2}{\srtype} &=
    \srtf{\sigma_1}{\srtype} \cap \srt{\sigma_2}{\srtype}
\end{align*}

\subsubsection{type system for negative types}

\infrule[F-App]{
  \fenv \vdash \psi_1 : \sigma_1 \rightarrow \sigma_2
  \andalso \fenv \vdash \psi_2 : \sigma_2
}{
  \fenv \vdash \psi_1\ \psi_2 : \sigma_2
}

\infrule[F-AppInt]{
  \fenv \vdash \psi_1 \COL \typeint{x} \rightarrow \sigma
}{
  \fenv \vdash \psi_1\ n \COL [n/x]\sigma
}

\infrule[F-Conj]{
  \fenv \vdash \psi_1 : \typeboolf{\theta_1}
  \andalso \fenv \vdash \psi_2 \COL \typeboolf{\theta_2}
}{
  \fenv \vdash \psi_1 \land \psi_2 \COL \typeboolf{\theta_1 \lor \theta_2}
}

\infrule[F-Disj]{
  \fenv \vdash \psi_1 : \typeboolf{\theta}
  \andalso \fenv \vdash \psi_2 \COL \typeboolf{\theta}
}{
  \fenv \vdash \psi_1 \lor \psi_2 \COL \typeboolf{\theta}
}

\infrule[F-Constr]{}{
  \fenv \vdash \theta \COL \typeboolf{\lnot \theta}
}

\infrule[F-Var]{}{
  \fenv, x \COL \sigma \vdash x \COL \sigma
}

\infrule[F-Forall]{
    \fenv, x \COL \stypeint \vdash \psi \typeboolf{\theta}
}{
  \fenv \vdash \forall x. \psi \COL \typeboolf{\exists x.\theta}
}

\infrule[F-Intersect-I]{
\fenv \vdash \psi \COL \sigma_1
\andalso \fenv \vdash \psi \COL \sigma_2
}{
\fenv \vdash \psi \COL \sigma_1 \intersect \sigma_2
}

\infrule[F-Intersect-E1]{
\fenv \vdash \psi \COL \sigma_1 \intersect \sigma_2
}{
\fenv \vdash \psi \COL \sigma_1
}

\infrule[F-Intersect-E2]{
\fenv \vdash \psi \COL \sigma_1 \intersect \sigma_2
}{
\fenv \vdash \psi \COL \sigma_2
}

\infrule[F-Subsumption]{
\fenv \vdash \psi \COL \sigma_1
\andalso \fenv; \true \vdash \sigma_1 \prec \sigma_2
}{
\fenv \vdash \psi \COL \sigma_2
}



The type system for positive types are almost the same as the one by Katsura et al., but we have to introduce a rule for intersection types.

\infrule[P-Intersect-I]{
\Gamma \vdash \psi \COL \tau_1
\andalso \Gamma \vdash \psi \COL \tau_2
}{
\Gamma \vdash \psi \COL \tau_1 \intersect \tau_2
}


\infrule[P-Intersect-E1]{
\Gamma \vdash \psi \COL \tau_1 \intersect \tau_2
}{
\Gamma \vdash \psi \COL \tau_1
}

\infrule[P-Intersect-E2]{
\Gamma \vdash \psi \COL \tau_1 \intersect \tau_2
}{
\Gamma \vdash \psi \COL \tau_2
}

Also, we introduce subtyping rules.

\infrule[F-Prop]{
\fenv \models \Theta \land \theta_2 \Rightarrow \theta_1
}{
\fenv; \Theta \vdash \typeboolf{\theta_1} \prec \typeboolf{\theta_2}
}
\infrule[F-Int-Predicate]{
\fenv, \typeint{x};\Theta \models \sigma_1 \prec \sigma_2
}{
\fenv; \Theta \vdash \typeint{x} \to \sigma_1 \prec \typeint{x} \to \sigma_2
}
\infrule[F-HO-Predicate]{
\fenv; \Theta \land rty(\sigma_2') \models \sigma_2 \prec \sigma_1
\andalso \fenv; \Theta \models \sigma_1' \prec \sigma_2'
}{
\fenv; \Theta \vdash \sigma_1 \to \sigma_1' \prec \sigma_2 \to \sigma_2'
}

\infrule[F-Intersection]{
\forall i \in \{1,2\}. \exists j \in \{1,2\}. \fenv; \Theta \vdash \sigma_i \prec \sigma_j
}{
\fenv; \Theta \vdash \sigma_1 \intersect \sigma_1' \prec \sigma_2 \intersect \sigma_2'
}

\begin{lemma}[Soundness of the type system]
    TBD
\end{lemma}
\begin{proof}
    TBD
\end{proof}


\subsection{HoPDR}

\subsubsection{State}

PDR hold two states: candidates and approximations \( M \mid \mid A \).

Candidates are represented in a labeled tree.
Each node is labeled with a tuple of negative type
judgement and step index
\( \langle X \COL \sigma, i \rangle \). The root of a candidate tree always has
\( \epsilon \) tag for the simplicity.
Let \( \epsilon \) be a tree which holds only \( \epsilon \) node
and we write \( T[L \bullet \{C_1\ C_2\cdots C_n\}] \) to mean \( T \) has a node with label \( L \) and its children are \( C_1, \cdots C_n \).

approximations are a sequence of refinement type environments \( \Gamma_1, \Gamma_2, \cdots, \Gamma_N \).
Note that type environments admit multiple items of the same name: \( x: \tau_1, x: \tau_2 \), meaning that the intersection type \( x: \tau_1 \wedge \tau_2 \).

\subsubsection{Conflict relation}

We define two maps: \( \conflict \) and \( \consistent \). They map a negative type and a positive type to a constraint formula. We use infix style notation. That is we write \( \sigma \conflict \tau \) to denote \( (\conflict)(\sigma, \tau) \).
Then we define the maps as follows:
\begin{align*}
    \typeboolf{\theta_1} \consistent \typebool{\theta_2}
        &= \lnot(\theta_1 \land \theta_2) \\
    \typeint{x} \to \sigma \consistent \typeint{x} \to \tau
        &= \forall x. \sigma \consistent \tau \\
    \sigma_1 \to \sigma_2 \consistent \tau_1 \to \tau_2
        &= \lnot(\sigma_1 \consistent \tau_1)
            \lor \sigma_2 \consistent \tau_2
\end{align*}
We define \( \sigma \conflict \tau = \lnot(\sigma \consistent \tau) \).

Intuitively, \( \Gamma \vdash \sigma \conflict \tau \) means that
there is no formula which have both \( \sigma \) and \( \tau \) type; \( \Gamma \vdash \sigma \consistent \tau \) is the dual of it.


\begin{lemma}[Soundness of conflict]
\label{lemma:conflict}
For any type environment \( \Gamma \), simple type \( \rho \), negative type \( \vdash \sigma :: \rho \) and refinement type \( \vdash \tau :: \rho \), the followings hold:
\begin{itemize}
    \item \( \Gamma \vdash \sigma \conflict \tau \Rightarrow
    \exists \alpha \in \val{\Gamma}.
    \semtf{\Gamma \vdash \sigma}(\alpha) \cap \semt{\Gamma \vdash \tau}(\alpha) = \emptyset \)

    \item \( \Gamma \vdash \sigma \consistent \tau \Rightarrow
    \forall \alpha \in \val{\Gamma}.
    \semtf{\Gamma \vdash \sigma}(\alpha) \cap \semt{\Gamma \vdash \tau}(\alpha) \neq \emptyset \).
\end{itemize}
\end{lemma}

\begin{proof}
By induction on the structure of the derivation.

Case \( \sigma_1 \to \sigma_2 \conflict \tau_1 \to \tau_2 \)
and \( \sigma_1' \to \sigma_2' \consistent \tau_1' \to \tau_2' \):
First, we consider the conflict one. That is, we assume \( \Gamma \vdash \sigma_1 \to \sigma_2 \conflict \tau_1 \to \tau_2 \).
From the induction hypothesis, there exists \( \alpha \in \val{\Gamma} \) such that \( \semtf{\sigma_1}(\alpha) \cap \semt{\tau_1}(\alpha) \neq \emptyset \).
We take \( g \in  \semtf{\sigma_1}(\alpha) \cap \semt{\tau_1}(\alpha) \). Then, we arbitrarily take \( f \in \semtf{\sigma_1 \to \sigma_2}(\alpha) \).
From the definition of the semantics of negative types, we have \( f(g) \in \semt{\sigma_2}(\alpha) \).
However, from the induction hypothesis, \( \semtf{\sigma_2}(\alpha) \cap \semt{\tau_2}(\alpha) = \emptyset \) holds, so
\( f(g) \not \in \semt{\tau_2}(\alpha) \).
Therefore, \( f \not \in \semt{\tau_1 \to \tau_2}(\alpha \) holds.
Since \( f \) was taken arbitrarily, \( \semtf{\sigma_1 \to \sigma_2}(\alpha) \cap \semt{\tau_1 \to \tau_2}(\alpha) = \emptyset \).

Secondly, we consider the consistent one. This can be proved in a similar manner. (TBD)
\end{proof}


For a type environment \( \Gamma \), a negative type \( \sigma \) and a variable \( X \), we write \( X : \sigma \consistent \Gamma \) for \( \forall X \COL \tau \in \Gamma. \vdash \sigma \consistent \tau \) and \(X : \sigma \conflict \Gamma\)
for \( \exists X \COL \tau \in \Gamma. \vdash \sigma \conflict \tau \).

\subsubsection{Algorithm Specification}

The input of this algorithm is definite formulas \(\{F_i\}_i\) and a goal formula
\(\psi\).

Then, let \( \Gamma_\true \) be \( \{F_i: \tau_\top\}_i \)
and \( \Gamma_\false \) be \( \{F_i: \tau_\bot \}_i \).
We write \( \Gamma \vdash \Gamma' \) when for any \( F \COL \tau \in \Gamma' \)
and \( \Gamma \vdash \psi_F \COL \tau \).
Also, let \( N \) be the length of approximations.

\begin{align*}
    \textbf{Initialize} &\quad
        \Longrightarrow
        \epsilon \mid \mid [\Gamma_0 \leftarrow \Gamma_\true]
        &&\\
    \textbf{Valid} &\quad
        \epsilon \mid \mid A \Longrightarrow \textit{Valid}
        &&\quad \textbf{if} \ \Gamma_{i} \vdash \Gamma_{i-1} \\
        &&&\quad \quad \Gamma_i \vdash \psi \COL \typebool{\true}
        \\
    \textbf{Unfold} &\quad
        \epsilon \mid \mid A \Longrightarrow
        \epsilon \mid \mid A[\Gamma_{N+1} \leftarrow \Gamma_\false]
        &&\quad \textbf{if} \ \Gamma_{N} \vdash \psi \COL \typebool{\true}
        \\
    \textbf{Candidate} &\quad
        \epsilon \mid \mid A
        \Longrightarrow&&\\
        & \epsilon[\epsilon \bullet \langle X_1 \COL \sigma_1, N \rangle, \cdots, \langle X_n \COL \sigma_n, N \rangle]\mid \mid A
        && \quad \textbf{if} \ \Gamma_N \not \vdash \psi \COL
        \typebool{\true},\\
        &&&\quad  X_1 \COL \sigma_1 , \cdots, X_n \COL \sigma_n\\
        &&&\quad \quad \quad=\textbf{CEX}(\Gamma_N, \psi, \typeboolf{\true})
        \\
    \textbf{Conflict} &\quad
        T[L \bullet \langle X_1: \sigma_1, i + 1 \rangle, \cdots,
                     \langle X_n: \sigma_n, i + 1 \rangle
        ] \mid \mid A
        \Longrightarrow&&\\
        & T[L]
        \mid \mid A[\Gamma_j \leftarrow \Gamma_j \cup \{X_k: \tau'\}]_{j=1}^{i+1}
        &&\quad \textbf{if} \  \exists k \in [1, n].
        \sigma_k \conflict \tau' \\
        &&&\quad \quad \text{ and } X_k \COL \tau' \in F(\Gamma_i)
        \\
    \textbf{Decide} &\quad
         T[\langle X: \sigma, i + 1 \rangle], \cdots\mid \mid A
        \Longrightarrow&&\\
        &  T[\langle X: \sigma, i + 1 \rangle
                \bullet \langle X_1 \COL \sigma_1, i\rangle, \cdots,
                        \langle X_n \COL \sigma_n, i\rangle]
        \mid \mid A
        &&\quad \textbf{if} \ X \COL \sigma \consistent F(\Gamma_i),\\
        &&&\quad  X_1 \COL \sigma_1, \cdots, X_n \COL \sigma_n\\
        &&&\quad\quad\quad= \textbf{CEX}(\Gamma_i, \psi_X, \sigma)
        \\
    \textbf{Model} &\quad
        T \mid \mid A \Rightarrow Invalid
        &&\quad \textbf{if} \text{ all the leaves of } T \text{ are } \langle \_, 0 \rangle \
        \\
\end{align*}


\subsubsection{\( \mathbf{CEX} \)}

\( \textbf{CEX}(\Gamma, \psi, \sigma) \) is a type inference routine of negative types.
Given \( \Gamma \), \( \psi \) and \( \sigma \),
\( \textbf{CEX}(\Gamma, \psi, \sigma) \) returns a negative type environment \( \fenv \) such that
\( \Gamma \sim \fenv \) and \( \fenv \vdash \psi \COL \sigma \).

\subsubsection{\( F\)}

We define \( F \) as follows:
\begin{align*}
    F(\Gamma) = \{ F_i \COL \tau \mid
        \Gamma \vdash \psi_{F_i} \COL \tau \}.
\end{align*}

%\subsection{How to achieve \( \mathbf{CEX}  \) and \( F \) using SMT}
%
%\subsubsection{ dual operation }
%
%For a negative type \( \sigma \), we define a positive type \( \dual{\sigma} \)
%as follows:
%\begin{align*}
%    \dual{\typeboolf{\theta}} &= \typebool{\lnot \theta}\\
%    \dual{\typeint{x} \to \sigma} &= \typeint{x} \to \dual{\sigma}\\
%    \dual{\sigma_1 \to \sigma_2} &= \dual{\sigma_1} \to \dual{\sigma_2}\\
%    \dual{\sigma_1 \intersect \sigma_2} &= ???
%\end{align*}
%Note that \( \sigma \consistent \dual{\sigma} \) holds.
%
%\begin{lemma}
%For a negative type \( \sigma \),
%a positive type environment \( \Gamma \)
%and a clause \( X \),
%\( \Gamma \vdash \psi_X \COL \dual{\sigma} \) holds if and only if
%there exists \( \tau \in F(\Gamma) \) such that
%\( \sigma \conflict \tau \).
%\end{lemma}
%
%
%\begin{proof}
%\( \Rightarrow \) is trivial. So we consider \( \Leftarrow \) case.
%We prove this by induction of the structure of the negative type \( \sigma \).
%\begin{itemize}
%    \item Case \( \typeboolf{\theta} \): if the
%\end{itemize}
%\end{proof}
%
%From Lemma \ref{lemma:conflict}, the condition checking in \rconflict{} can be
%achieved through an SMT solver.

\subsubsection{Deciding whether Decide or Conflict is taken}

To decide which is taken, we have to find a refinement type \( \tau' \) such that \( \sigma_k \conflict \tau' \text{ and } X_k \COL \tau' \in F(\Gamma_i) \) holds.
To do this, we do a template based-approach to infer such types using SMT solvers.


Given a candidate \( X \COL \sigma \) and positive type environment \( \Gamma \), first we generate a following type template:
\begin{align*}
    \textbf{Template}(\tilde{x}, \typeboolf{\theta})
        &= \typebool{P(\tilde{x})} \quad \text{where } P \text{ is a fresh name} \\
    \textbf{Template}(\tilde{x}, \typeint{x} \to \sigma)
        &= \typeint{x} \to \textbf{Template}(\tilde{x} \cup \{x\}, \sigma)\\
    \textbf{Template}(\tilde{x}, \sigma_1 \to \sigma_2)
        &= \textbf{Template}(\tilde{x}, \sigma_1)
        \to \textbf{Template}(\tilde{x}, \sigma_2)
\end{align*}
Here, we think \( \tilde{x} \) is a sequence of variables.

Let \( \tau \) be \( \textbf{Template}(\{\}, \sigma) \) now.
Next, we generate constraints on conflict relation \( \sigma \conflict \tau \)
and derivation \( \Gamma \vdash \psi_X \COL \tau \).
The resulting constraint \( \Theta \) has the following form:
\begin{align*}
    \Theta &= P(\tilde{x})
            \mid \theta
            \mid \Theta_1 \land \Theta_2
            \mid \Theta_1 \lor \Theta_2
            \mid \lnot \Theta
            \mid \forall x. \Theta.
\end{align*}

Next, we replace predicate variables with templates.
We use the following template:
\begin{align*}
    &T(tag, \{x_0, \cdots, x_{n-1}\}) = \\
       &\quad \sum_{i=0}^{n-1} \alpha^{tag}_i x_i + \alpha^{tag} \geq 0
        \land \sum_{i=0}^{n-1} \beta^{tag}_i x_i + \beta^{tag} = 0
        \land \sum_{i=0}^{n-1} \gamma^{tag}_i x_i + \gamma^{tag} \neq 0
\end{align*}
For each predicate variable \( P \) whose arity is n, we create a new template
\( T(P, \{x_0, \cdots, x_{n-1}\}) \). We use variable name \( P \) as the tag for the variables appear in the template in order to avoid conflicting names between templates.

Finally, we consult an SMT solver whether this is satisfiable or not.
If it's sat, we can recover the refinement type, meaning that
we do \rconflict{}.
If it's unsat, we do \rdecide{}.

\subsubsection{Calculating \( \textbf{CEX} \) using SMT}

This is almost the same as the above discussion.
\( \textbf{CEX} \) can be seen as negative type inference. Given a positive type environment \( \Gamma \), a clause \( X \) and a negative type \( \sigma \), we want to find a negative type environment \( \Delta \) such that \( \Delta \consistent \Gamma \) and \( \Delta \vdash \psi_X \COL \sigma \).

To do this, we create a template for each clause \( X \in dom(\Gamma) \) in the same way as the above. Then from the consistent relation and derivation, we have constraints on the template, so using an SMT solver, we can infer the type environment.

\subsubsection{Current problem}

\begin{itemize}
    \item What shape of templates is better?
    \item What if we cannot find \( \tau \) nor \( \Delta \).
    In that case, we have to consider intersection types?
    \item When restricted to order-1 case, the above procedures is
    a little different from the original PDR. There is no interpolation.
    This perhaps leads to a divergence of the procedure.
\end{itemize}

\subsection{Optimization}

In the above procedure, every time CEX or a new positive type is generated, a new predicate will be introduced. This leads to the problem that the procedure will be divergent. To mitigate this problem, by introducing a flow analysis, we fix the set of possible refinement types for the arguments of each predicates and try \rconflict{} or \rdecide{} before the above procedure is done.

TODO

\subsection{Case order-2}

TODO

Try to investigate whether there is a better way to calculate the refinement types than template-based one.

%\subsection{Least formula of type \( \tau \)}
%
%We consider the following two operators:
%\begin{align*}
%    \psi \Downarrow \typebool{\constraint} \quad
%        &= \quad \psi \land \constraint
%        \\
%    \psi \Downarrow \typeint{x} \to \tau   \quad
%        &= \quad \lambda x. \psi \Downarrow \tau
%        \\
%    \psi \Downarrow \tau_1 \to \tau_2 \quad
%        &= \quad \lambda f. \psi \land f \in \tau_1 \Downarrow \tau_2
%        \\
%    \psi \Downarrow \tau_1 \intersect \tau_2 \to \ta\quad
%        &= TODO
%        \\
%    \psi \in \typebool{\constraint} \quad
%        &= \quad \lnot \theta \lor \psi
%        \\
%    \psi \in \typeint{x} \to \tau   \quad
%        &= \quad \forall x. \psi\ x \in \tau
%        \\
%    \psi \in \tau_1 \to \tau_2 \quad
%        &= \quad \psi\ (\true \Downarrow \tau_1) \in \tau_2
%        \\
%    \psi \in \tau_1 \intersect \tau_2 \quad
%        &= \psi \in \tau_1 \land \psi \in \tau_2
%        \\
%\end{align*}
%
%We abbreviate \( \true \Downarrow \tau_1 \) as \( \minimalize{\Gamma} \).
%Then, we define \( \minimalize{\Gamma} \) as \( \{ X \COL \minimalize{\Gamma(X)} \mid X \in dom(\Gamma) \} \).
%
%\begin{lemma}
%For any positive type environment \( \Gamma \), any formula \( \psi \) and a positive type \( \tau \), \( \Gamma \vdash \psi \COL \typebool{\theta} \) if and only if  \( \minimalize{\Gamma} \models \psi \COL \stypebool \).
%\end{lemma}

\subsection{Running examples 1. \( Sum \)}

\begin{align*}
    Sum\ x\ k\ &=_\nu (x > 0 \lor k\ x) \land
                    (x \leq 0 \lor Sum\ (x - 1) (\lambda r. k\ (x + r)))\\
    \psi &::= \forall x. Sum\ x\ (\lambda r. r \geq x)
\end{align*}

\begin{align*}
    &\Longrightarrow \rinitialize{} \\
    &\epsilon ||
    [\tau_0 = \typeint{x} \to (\typeint{r} \to \typebool{\false}) \to \typebool{\true}]\\
    &\Longrightarrow \runfold{} \\
    &\epsilon ||
    [\tau_0; \tau_1 = \typeint{x} \to (\typeint{r} \to \typebool{\true}) \to \typebool{\false}]\\
    &\Longrightarrow \rcandidate{}
    ; \sigma_1 = \typeint{x} \to (\typeint{r} \to \typeboolf{r < x}) \to \typeboolf{\true}  \\
    &
    [\sigma_1] ||
    [\tau_0; \tau_1 = \typeint{x} \to (\typeint{r} \to \typebool{\true}) \to \typebool{\false}]\\
    &\Longrightarrow \rconflict{};  F(\tau_0) \ni \typeint{x} \to (\typeint{r} \to \typebool{r \geq x} \to \typebool{\true}\\
    &
    [\sigma_1] ||
    [\tau_0; \tau_1 = \typeint{x} \to (\typeint{r} \to \typebool{r \geq x} \to \typebool{\true}]\\
    &\cdots
\end{align*}

\subsection{type checking}
\subsubsection{The problem of type checking}

In the middle of

Consider
\begin{align*}
    F\ x\ k =_\nu \forall y. y \neq x - 1 \lor
        & \{ (x > 0 \lor k\ x)\\
        & \land ( x \leq 0 \lor F\ y\ (L\ x\ k)) \}
\end{align*}
where \( L\ x\ k\ y =_\nu k\ (x + y) \).
We can prove the safety by the following type environment:

\begin{align*}
    F&: \typeint{x} \to (\typeint{y} \to \typebool{x \leq y}) \to
    \typebool{\true}\\
    L&: \typeint{x} \to (\typeint{z} \to \typebool{x \leq z}) \to
    \typeint{y}) \to \typebool{0 \leq y}
\end{align*}

Consider the following type environment \(\Gamma\)
\begin{align*}
    f&: (\typeint{x} \to \typebool{C_1}) \to ((\typeint{y} \to \typebool{C_2})
    \to \typebool{C_3})\\
    k&: \typeint{x} \to \typebool{C_4}\\
    g&: \typeint{x} \to \typebool{C_5}\\
    f'&: (\typeint{x} \to \typebool{C_1'}) \to ((\typeint{y} \to \typebool{C_2'})
    \to \typebool{C_3'})\\
    k'&: \typeint{x} \to \typebool{C_4'}\\
    g'&: \typeint{x} \to \typebool{C_5'}\\
    z&: \stypeint.
\end{align*}
We would like to prove \( \Gamma \vdash (f\ k\ g) \lor (f'\ k'\ g'):
\typebool{C}\)


\subsubsection{Algorithm of type checking}

\begin{algorithmic}
\STATE infer($\Gamma$, $f$) = ($\{\}$, $\Gamma(f)$)
\STATE infer($\Gamma$, $a\ x$) where $\Gamma(x) = \stypeint$ =
    \bindent
    \STATE let ($c, \typeint{y} \mapsto \tau$) = infer($\Gamma$, $a$) in
    \STATE ($[y/x]C, [y/x]\tau$)
    \eindent
\STATE infer($\Gamma$, $a_1\ a_2$) otherwise =
    \bindent
    \STATE let ($c_1, \sigma \mapsto \tau$) = infer($\Gamma$, $a_1$) in
    \STATE let ($c_2, \sigma'$) = infer($\Gamma$, $a_2$) in
    \STATE let ($\tau', L$) = generate\_template($\tau$) in
    \STATE let C = subtype($\sigma \mapsto \tau, \sigma'\mapsto \tau'$) in
    \STATE let C' = simplify($ C \cup C' \cup C'', L)$ in
    \STATE ($ C', \tau'$)
    \eindent
\STATE simplify(C, L) =
    \bindent
    \WHILE{let $P(\tilde{x}) \Leftarrow \psi$ where $P \not \in L$}
        \FORALL{$Q(\tilde{y}) \Leftarrow \psi' \land P(\tilde{x})$ in $C$}
            \STATE remove $P(\tilde{x}) \Leftarrow \psi$ from $C$
            \STATE add $Q(\tilde{y}) \Leftarrow \psi' \land \psi$ to $C$
        \ENDFOR
    \ENDWHILE
    \eindent

\end{algorithmic}

Consider the following type environment \(\Gamma\)
\begin{align*}
    f&: (\typeint{x} \to \typebool{C_1}) \to ((\typeint{x} \to \typebool{C_2})
    \to \typebool{C_3})\\
    g&: \typeint{x} \to \typebool{C_4}\\
    h&: (\typeint{x} \to \typebool{C_1'}) \to x \to \typebool{C_3'}\\
    i&: \typeint{x} \to \typebool{C_4'}\\
    y&: \stypeint
\end{align*}
Then, we try to check \(\Gamma \vdash (f\ g)\ (h\ i): \typebool{\true}\).
To do that, we first infer the type of this formula: infer\((\Gamma, (f\ g)\
(h\ i)\).
Observing that infer\((\Gamma, f\ g) = (C_1, (\typeint{x} \rightarrow
\typebool{P(x, y)}) \rightarrow \typebool{Q(y)} )\), infer\((\Gamma, f\ g)
= (C_2, \typeint{x} \rightarrow \typebool{R(x, y)})\),
generate\_template\((\typebool{Q(y)}) = (\typebool{Q'(y)}, \{Q'\})\)
where \(C_1\) is
\begin{align*}
    Q(y) \Rightarrow &C_3\\
    Q(y) \land \Rightarrow &P(x, y)\\
    Q(y) \land C_1  \Rightarrow &C_4
\end{align*}
and \(C_2\) is
\begin{align*}
    R(x, y) \Rightarrow &C_3'\\
    R(x, y) \land C_1'\Rightarrow &C_4',
\end{align*}
subtype\(( (\typeint{x} \to \typebool{P(x, y)}) \to \typebool{Q(y)} \prec
(\typeint{x} \to \typebool{R(x, y)}) \to \typebool{Q'(y)}) \) returns \(C_3\):
\begin{align*}
    Q'(y) \Rightarrow &Q(y)\\
    Q'(y) \land P(x, y) \Rightarrow &C_4'.
\end{align*}
Then, simplify\((C_1 \cup C_2 \cup C_3, \{Q'\})\) will be
\begin{align*}
    Q'(y) \Rightarrow &C_3\\
    Q'(y) \land C_2 \Rightarrow &C_3'\\
    Q'(y) \land C_2 \land C_1' \Rightarrow &C_4'\\
    Q'(y) \land C_1 \Rightarrow &C_4.
\end{align*}

%
%
%\subsection{non-linear extension}
%
%GPDRの論文で言う、linearなCHCは、APLAS20の論文でtractableと読んでいたものに対応。今回は簡単のため、non-linear
%
%counterexampleは、ラベル付き木を成すものとする。ラベルは、型環境と展開数のタプルであり、空の木を \( \epsilon \)、木Tの中のノードNを、その子供のノード\( C_1, \cdots C_n \) を用いて、\( T[N \bullet C_1\ C_2\cdots C_n] \) と表す
%
%
%次にnon-linearなものを扱えるようにするために、PDR実行中に保持する状態を
%現在の最大の展開数を \( N \)としたときに、counterexampleの列と
%\( \Gamma_1 \cdots \Gamma_N \)のタプルとする。
%
%つまり、単一のcounterexample(列)から、複数のcounterexample(木)となる。
%これは、toplevelの述語もnon-linearになりうるので、その場合複数の木が生える。
%
%non-linearの拡張として一部のルールを以下のように変更
%
%\begin{align*}
%    \textbf{Candidate} &\quad
%        \epsilon \mid \mid A
%        \Longrightarrow
%        \langle \Gamma_1^N, \cdots \Gamma_n^N, N \rangle \mid \mid A
%        && \quad \textbf{if} \ \Gamma_N \not \vdash \psi \COL
%        \typebool{\true},\\
%        &&&\quad \Gamma_1^N, \cdots \Gamma_n^N = \textbf{CEX}(\Gamma, \psi, \typebool{\true})
%        \\
%    \textbf{Conflict} &\quad
%        T[\langle \Gamma \cup \{X: \tau\}, i + 1 \rangle], \cdots \mid \mid A
%        \Longrightarrow&&\\
%        & T[\langle \Gamma, i + 1 \rangle], \cdots
%        \mid \mid A[\Gamma_j \leftarrow \Gamma_j \cup \{X: \tau'\}]_{j=1}^{i+1}
%        &&\quad \textbf{if} \  \Gamma_i \vdash \psi_X \COL \tau' \text{ and }
%        \Gamma_i \vdash \tau' \prec \tau
%        \\
%    \textbf{Decide} &\quad
%         T[\langle\Gamma \cup \{X: \tau\}, i + 1 \rangle],  \cdots\mid \mid A
%        \Longrightarrow&&\\
%        &  T[\langle\Gamma \cup \{X: \tau\}, i + 1 \rangle \bullet \langle\Gamma_1^i, i\rangle, \cdots], \cdots
%        \mid \mid A
%        &&\quad \textbf{if} \ \Gamma_i \not\vdash \psi_X \COL \tau,\\
%        &&&\quad \Gamma_1^i, \cdots, \Gamma_n^i = \textbf{CEX}(\Gamma_i, \psi_X, \tau)
%        \\
%    \textbf{Model} &\quad
%        \langle T, \cdots \rangle M \mid \mid A \Rightarrow Invalid
%        &&\quad \textbf{if Tの全ての葉が展開数0になったとき} \
%        \\
%\end{align*}
%
%
%\subsection{non-linear extension example}
%
%\begin{align*}
%    F(x) &= x \neq 0 \wedge F\ (x+1) \\
%    G(x) &= x \neq 0 \wedge G\ (x-1)  \\
%    M    &= \forall x. x=0 \lor F\ x \lor G\ x
%\end{align*}
%
%\begin{itemize}
%    \item Initialize: \(\Gamma_0 = F(\{X \COL x \rightarrow \top\}) = \{ X \COL
%        \typeint{x} \to \typebool{x \neq 0} \} \)
%    \item Unfold: \(\Gamma_0, \Gamma_1 = \{X \COL x \rightarrow \bot\}\)
%\end{itemize}

\end{document}

